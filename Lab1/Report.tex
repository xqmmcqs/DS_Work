\documentclass{article}

\usepackage{ctex}
\usepackage[top=0.7in,bottom=0.7in,left=0.5in,right=0.5in]{geometry}
\usepackage{array}
\usepackage{multirow}
\usepackage{graphicx}
\usepackage{fancyhdr}
\usepackage{lastpage}
\usepackage{extramarks}
\usepackage{amsmath}
\usepackage{listings}
\usepackage{fontspec}
\newfontfamily\menlo{Menlo}
\usepackage{xcolor} % 定制颜色
\definecolor{mygreen}{rgb}{0,0.6,0}
\definecolor{mygray}{rgb}{0.5,0.5,0.5}
\definecolor{mymauve}{rgb}{0.58,0,0.82}
\lstset{ %
backgroundcolor=\color{white},      % choose the background color
basicstyle=\footnotesize\ttfamily,  % size of fonts used for the code
columns=fullflexible,
tabsize=4,
breaklines=true,               % automatic line breaking only at whitespace
captionpos=b,                  % sets the caption-position to bottom
commentstyle=\color{mygreen},  % comment style
escapeinside={\%*}{*)},        % if you want to add LaTeX within your code
keywordstyle=\color{blue},     % keyword style
stringstyle=\color{mymauve}\ttfamily,  % string literal style
frame=single,
rulesepcolor=\color{red!20!green!20!blue!20},
% identifierstyle=\color{red},
language=c++,
}

\newcommand{\hmwkTitle}{加里森的任务\ 实验报告}
\newcommand{\hmwkClass}{数据结构}
\newcommand{\hmwkClassInstructor}{}
\newcommand{\hmwkAuthorName}{毛子恒\ 李臻\ 张梓靖}

\pagestyle{fancy}
\lhead{\hmwkAuthorName}
\chead{\hmwkClass\ : \hmwkTitle}
\rhead{\firstxmark}
\lfoot{\lastxmark}
\cfoot{\thepage}
\renewcommand\headrulewidth{0.4pt}
\renewcommand\footrulewidth{0.4pt}

\title{\hmwkClass\ :\hmwkTitle}
\author{\hmwkAuthorName}

\newcommand{\True}{\textbf{T}}
\newcommand{\False}{\textbf{F}}

\begin{document}

\maketitle

\section*{小组成员}

\setlength{\tabcolsep}{9mm}
{
    \begin{table}[htbp]
        \centering
        \begin{tabular}{llll}
            班级:2019211309 & 姓名:毛子恒 & 学号:2019211397 & 分工:代码\ 文档   \\
            
            班级:2019211310 & 姓名:李臻   & 学号:2019211458 & 分工:测试\ 文档   \\
            
            班级:2019211308 & 姓名:张梓靖 & 学号:2019211379 & 分工:可视化\ 文档 \\
        \end{tabular}
    \end{table}
}

\section{需求分析}

\subsection{题目描述}

在由序号为$1$至$n$的$n$个元素依次排列并且首尾相接而组成的环中,规定初始时从序号$1$开始依次经过$2,3,...$元素走到第$n$个元素的方向为正方向。

初始时以第$x$个元素为起点$st$,重复以下过程$n-1$次:以$st$为第1个元素,沿正方向找到第$y$个元素$del$,从环中删除$del$元素,再将原$del$的下一个元素作为新的$st$。

求经过$n-1$次操作之后,环中仅剩的一个元素的序号是否是1。

\subsection{输入描述}

程序从标准输入中读入数据。输入一行三个整数,用空格分隔,分别表示$n,x,y$。

其中各个值的范围需要满足$1 < n \leq 10^9\quad 0 < x \leq n\quad 0 < y \leq 10^9$。

由于程序时间复杂度较大,建议$n \leq 10^4$。

\subsection{输出描述}

输出分为三种情况:

\begin{enumerate}
    \item 输入合法,程序正常运行结束。此时输出两行,第一行一个字符串"Yes"或者"No"(不带引号),分别表示最后一个元素是/不是1,第二行一个数字,表示最后一个元素的序号。
    \item 输入不合法。此时输出一行一个字符串"Please check your input."(不带引号)。
    \item 程序发生运行时错误,比如内存分配失败。此时程序没有输出。
\end{enumerate}

\subsection{样例输入输出}

\subsubsection{样例输入输出1}

【输入】

\begin{lstlisting}[language={bash},
    basicstyle=\small\menlo]
10 1 3
\end{lstlisting}

【输出】

\begin{lstlisting}[language={bash},
    basicstyle=\small\menlo]
No
4
\end{lstlisting}

\subsubsection{样例输入输出2}

【输入】

\begin{lstlisting}[language={bash},
    basicstyle=\small\menlo]
10 3 7
\end{lstlisting}

【输出】

\begin{lstlisting}[language={bash},
    basicstyle=\small\menlo]
Yes
1
\end{lstlisting}

\subsubsection{样例输入输出3}

【输入】

\begin{lstlisting}[language={bash},
    basicstyle=\small\menlo]
100 87 305
\end{lstlisting}

【输出】

\begin{lstlisting}[language={bash},
    basicstyle=\small\menlo]
No
50
\end{lstlisting}

\subsubsection{样例输入输出4}

【输入】

\begin{lstlisting}[language={bash},
    basicstyle=\small\menlo]
1000 725 801
\end{lstlisting}

【输出】

\begin{lstlisting}[language={bash},
    basicstyle=\small\menlo]
No
798
\end{lstlisting}

\subsubsection{样例输入输出5}

【输入】

\begin{lstlisting}[language={bash},
    basicstyle=\small\menlo]
1 1 3
\end{lstlisting}

【输出】

\begin{lstlisting}[language={bash},
    basicstyle=\small\menlo]
Please check your input.
\end{lstlisting}

\subsubsection{样例输入输出6}

【输入】

\begin{lstlisting}[language={bash},
    basicstyle=\small\menlo]
5 6 3
\end{lstlisting}

【输出】

\begin{lstlisting}[language={bash},
    basicstyle=\small\menlo]
Please check your input.
\end{lstlisting}

\subsection{程序功能}

程序通过给定的$n,x,y$计算出最后环中仅剩的元素序号,并且与1比较。

\section{概要设计}

\subsection{问题解决的思路}

使用单循环链表维护此约瑟夫环,首先在链表中依次插入$n$个结点表示$n$名队员,以$now$指针模拟计数过程。

从头结点找到第$x$个结点,此后执行以下操作$n-1$次:找到当前结点之后的第$y-1$个结点,删除这个结点。

此题中单循环链表实现了初始化、判空、在指定位置增加节点、删除指定位置的节点、释放空间这五种操作。

由于链表的删除操作实现是删除给定结点的后继,所以$now$指针始终指向当前正在计数元素的前驱。

由于单循环链表中存在一个特殊的头结点,所以另实现一个函数,返回某个结点的后继(跳过头结点)。

更多细节在调试分析报告部分中讨论。

\subsection{链表的定义}

\begin{lstlisting}[language={C},
    numbers=left,
    numberstyle=\tiny\menlo,
    basicstyle=\small\menlo]
// 数据对象
typedef struct node
{
    int item;
    struct Node * next;
} Node;

typedef Node * List;

// 基本操作
/*
 * 操作:初始化链表
 * 后件:plist指向一个循环链表的头结点
 */
void initList(List * plist);

/*
 * 操作:判断链表是否为空
 * 前件:list是循环链表的头结点
 * 后件:如果该链表为空,返回true,否则返回false
 */
bool isEmpty(const List list);

/*
 * 操作:向链表的某个节点后插入一个节点
 * 前件:pnode是链表中的某一个节点
 * 后件:如果成功,pnode之后添加一个新节点,item属性为传入的第二个参数
 */
void addNode(List pnode, int item);

/*
 * 操作:删除链表中指定的节点
 * 前件:list是该链表的头结点,pnode是需要删除的节点
 * 后件:删除链表中的pnode节点
 */
void delNode(List list, List pnode);

/*
 * 操作:找到链表中某一节点的后继
 * 前件:pnode指向链表中的某一个节点
 * 后件:函数返回pnode的后继,并且跳过头结点
 */
List nextNode(const List pnode);

/*
 * 操作:释放链表空间
 * 前件:plist指向需要释放空间的链表的头结点
 * 后件:释放plist指向链表的空间,plist重置为空指针
 */
void destroyList(List * plist);
\end{lstlisting}

\subsection{主程序的流程}

\begin{enumerate}
    \item 输入
    \item 初始化链表
    \item 在链表中依次插入$n$个结点
    \item 找到第$x$个结点
    \item 循环$n-1$次:找到当前节点之后的第$y-1$个结点,删除这个结点
    \item 输出
    \item 释放空间
\end{enumerate}

\subsection{各程序模块之间的层次关系}

\section{详细设计}

\subsection{链表的实现}

\subsection{函数的调用关系图}

\section{调试分析报告}

\subsection{调试过程中遇到的问题和解决方案}

\subsection{设计实现的回顾讨论}

\subsection{算法复杂度分析}

\subsection{改进设想的经验和体会}

\subsubsection{改进1}

在主程序的这一部分:

\begin{lstlisting}[language={C},
    numbers=left,
    numberstyle=\tiny\menlo,
    basicstyle=\small\menlo]
for (int i = 1; i <= n; ++i) // 逐个添加元素
{
    addNode(now, i);
    now = nextNode(now);
}
now = list;
for (int i = 1; i < x; ++i) // 找到第x个元素的前驱
    now = nextNode(now);
\end{lstlisting}

可以另用一个指针变量在向链表逐个添加元素的同时记录第$x-1$个元素的位置,以省去第二个循环。优化后的实现如下:

\begin{lstlisting}[language={C},
    numbers=left,
    numberstyle=\tiny\menlo,
    basicstyle=\small\menlo]
List temp = NULL;
for (int i = 1; i <= n; ++i)
{
    addNode(now, i);
    now = nextNode(now);
    if (i == x - 1) temp = now;
}
now = temp;
\end{lstlisting}

\subsubsection{改进2}

在主程序的这一部分:

\begin{lstlisting}[language={C},
    numbers=left,
    numberstyle=\tiny\menlo,
    basicstyle=\small\menlo]
for (int i = 1; i < n; ++i)
{
    for (int j = 1; j < y; ++j)
        now = nextNode(now);
    delNode(list, now);
}
\end{lstlisting}

对于有$n-i+1$个元素的环,找到当前元素之后的第$y-1$个元素和找到当前元素之后的第$(y-1)\bmod (n-i+1)$个元素并无区别。优化后的实现如下:

\begin{lstlisting}[language={C},
    numbers=left,
    numberstyle=\tiny\menlo,
    basicstyle=\small\menlo]
for (int i = 1; i < n; ++i)
{
    for (int j = 1; j <= (y - 1) % (n - i + 1); ++j)
        now = nextNode(now);
    delNode(list, now);
}
\end{lstlisting}

当$y$比$n$大的时候对时间复杂度有很可观的优化。

\section{用户使用说明}

\section{测试结果}

\subsection{测试实例1}

\end{document}